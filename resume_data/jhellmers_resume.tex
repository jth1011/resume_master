%-------------------------
% Resume in Latex
% Author : Jackson Hellmers
% License : MIT
%------------------------

\documentclass[letterpaper,11pt]{article}

\usepackage{latexsym}
\usepackage[empty]{fullpage}
\usepackage{titlesec}
\usepackage{marvosym}
\usepackage[usenames,dvipsnames]{color}
\usepackage{verbatim}
\usepackage{enumitem}
\usepackage[hidelinks]{hyperref}
\usepackage{fancyhdr}
\usepackage[english]{babel}
\usepackage{tabularx}
\usepackage[T1]{fontenc}

\pagestyle{fancy}
\fancyhf{} % clear all header and footer fields
\fancyfoot{}
\renewcommand{\headrulewidth}{0pt}
\renewcommand{\footrulewidth}{0pt}

% Adjust margins
\addtolength{\oddsidemargin}{-0.5in}
\addtolength{\evensidemargin}{-0.5in}
\addtolength{\textwidth}{1in}
\addtolength{\topmargin}{-.5in}
\addtolength{\textheight}{1.0in}

\urlstyle{same}

\raggedbottom
\raggedright
\setlength{\tabcolsep}{0in}

% Sections formatting
\titleformat{\section}{
  \vspace{-4pt}\scshape\raggedright\large
}{}{0em}{}[\color{black}\titlerule \vspace{-5pt}]

%-------------------------
% Custom commands
\newcommand{\resumeItem}[2]{
  \item\small{
    \textbf{#1}{: #2 \vspace{-2pt}}
  }
}

\newcommand{\resumeSubheading}[4]{
  \vspace{-1pt}\item
    \begin{tabular*}{0.97\textwidth}[t]{l@{\extracolsep{\fill}}r}
      \textbf{#1} & #2 \\
      \textit{\small#3} & \textit{\small #4} \\
    \end{tabular*}\vspace{-5pt}
}

\newcommand{\resumeSubSubheading}[2]{
    \begin{tabular*}{0.97\textwidth}{l@{\extracolsep{\fill}}r}
      \textit{\small#1} & \textit{\small #2} \\
    \end{tabular*}\vspace{-5pt}
}

\newcommand{\resumeSubItem}[2]{\resumeItem{#1}{#2}\vspace{-4pt}}

\renewcommand{\labelitemii}{$\circ$}

\newcommand{\resumeSubHeadingListStart}{\begin{itemize}[leftmargin=*,label={}]}
\newcommand{\resumeSubHeadingListEnd}{\end{itemize}}
\newcommand{\resumeItemListStart}{\begin{itemize}}
\newcommand{\resumeItemListEnd}{\end{itemize}\vspace{-5pt}}


%------------
% my new template


%-------------------------------------------
%%%%%%  CV STARTS HERE  %%%%%%%%%%%%%%%%%%%%%%%%%%%%
\newcommand{\exptitle}[4]{
  \vspace{7pt}
  \begin{tabular*}{1.00\textwidth}[t]{l@{\extracolsep{\fill}}r}
    \textbf{#1}, #2, \textit{#3} & #4 \\
  \end{tabular*}\vspace{-5pt}
}

\newcommand{\projtitle}[2]{
  \vspace{7pt}
  \begin{tabular*}{1.00\textwidth}[t]{l@{\extracolsep{\fill}}r}
    \textbf{#1}, #2 \\
  \end{tabular*}\vspace{-5pt}
}

\newcommand{\award}[2]{
  \vspace{7pt}
  \begin{tabular*}{1.00\textwidth}[t]{l@{\extracolsep{\fill}}r}
    \textbf{#1} & #2 \\
  \end{tabular*}
}



\newcommand{\expstart}{\begin{itemize}[leftmargin=5mm]}
\newcommand{\expend}{\end{itemize}\vspace{-5pt}}
\newcommand{\expitem}[1]{\item\small{{#1 \vspace{-5pt}}}}

\begin{document}

\begin{tabular*}{\textwidth}{l@{\extracolsep{\fill}}r}
    \textbf{\Large Jackson Hellmers} & \href{mailto:hellmejt.hellmers@gmail.com}{hellmejt.hellmers@gmail.com} \\
    \small{\href{https://github.com/jth1011}{ GitHub: \bf jth1011} \ \ \href{https://www.linkedin.com/in/jackson-hellmers/}{LinkedIn: \bf jackson-hellmers}} & \href{tel:16128048090}{(612)804-8090} \\
\end{tabular*}

%-----------EDUCATION-----------------
\section{Education}
  \vspace*{1pt}
   \begin{tabular*}{1.00\textwidth}[t]{l@{\extracolsep{\fill}}r}
    \textbf{Master of Science in Electrical Engineering} & {GPA: Undetermined} \\
    \small {University of Wisconsin-Madison} & \small{Sep 2021 - Present} \vspace*{3pt}
  \end{tabular*}
  \begin{tabular*}{1.00\textwidth}[t]{l@{\extracolsep{\fill}}r}
    \textbf{Bachelor of Science in Electrical and Computer Engineering} & {GPA: 3.95/4.00} \\
    \small{University of Wisconsin-Madison} & \small{Sep 2017 - May 2021}
    \vspace*{6pt}
  \end{tabular*}
\small{\textbf{Highlighted Coursework}{: Graduate Machine Learning, Artificial Intelligence, Neural Networks,\\ Image Processing, Computer Vision, Probabilistic Modeling, Signal Processing}}

%
%--------PROGRAMMING SKILLS------------
\section{Programming Skills}

\textbf{Languages}: Python, Java, C/C++, MATLAB, SQL  \\
\textbf{Machine Learning Libraries}: Tensorflow, PyTorch, Keras, Scikit-learn, Numpy, Pandas, Matplotlib, OpenCV \vspace{-2pt}

%-----------EXPERIENCE-----------------
\section{Work 
Experience}
 \vspace{-6pt}
  \exptitle{Electrical Engineer Intern}{Innovative Signal Analysis}{Richardson, TX}{May 2021 - Aug 2021}
  \expstart
    \expitem{Programmed Xilinx UltraScale RFSoC boards using Vivado Design Suite and Vitis IDE.}
    \expitem{Generated bare metal applications to benchmark Real-time Computing cores.}
    \expitem{Designed Verilog modules to measure latency between a system’s CPU and FPGA.}
    \expitem{Created PCBs to measure the power draw of various components using fixed-voltage supply \\ and current sense ICs.}
  \expend
  
  \exptitle{Design Verification Test Engineer}{Extreme Engineering Solutions}{Madison, WI}{Jan 2020 - Sep 2020}
  \expstart
    \expitem{Worked with high-bandwidth (50 GHz) oscilloscopes to verify clock and data signal functionality}
    \expitem{Constructed and improved PCB designs to adjust clock slew rates and power supply switching speeds.}
    \expitem{Recorded daily activities and software adjustments through online tickets and version \\ control applications.}
    \expitem{Documented verification tests measuring the compliance of numerous I/O interfaces such as \\ USB 3.0, 100 Gig Ethernet and PCIe 4.0.}
  \expend
  
  \exptitle{Systems Development Intern}{Exact Sciences}{Madison, WI}{May 2019 - Dec 2019}
  \expstart
    \expitem{Designed and prototyped simple DC motor control and voltage regulation circuits.}
    \expitem{Developed and ran autonomous data-driven C++ programs on Arduino \\ boards and Python programs on Raspberry Pi boards.}
    \expitem{Followed wiring diagrams and guidelines to assemble industrial sized automated systems.}
  \expend

%-----------PROJECTS-----------------
\section{Projects}
  \vspace{-6pt}
  \projtitle{\href{https://github.com/jth1011/Digit-Recognizer}{Digit Recognition}}{Personal Project}
  \expstart
    \expitem{Used \textbf{Scikit-learn} and \textbf{Tensorflow} python libraries to create numerous digit recognition models (KNN, Naive Bayes, CNN).}
    \expitem{Uploaded the most successful model to Kaggle's MNIST Classification Challenge and was placed in top 5\%.}
  \expend
  
    \projtitle{\href{https://github.com/jth1011/Playing-Card-Identifier}{Playing Card Identifier}}{Personal Project}
  \expstart
    \expitem{Created a CNN to properly identify the value and suit of a playing card within an image.}
    \expitem{Used the Image Augmentation library in \textbf{Keras} to generate a large set of training images from a small set of actual images and prevent overfitting.}
    \expitem{Created training and testing datasets from scratch and preprocessed images using \textbf{OpenCV}.}
  \expend
 
  \projtitle{\href{https://github.com/jth1011}{CS760 Final Project - DeVise}}{Academic Project}
  \expstart
    \expitem{Inspired by and adapted from \href{http://www.cs.toronto.edu/~ranzato/publications/frome_nips2013.pdf}{\textbf{Google Research Paper:}}}DeVise - A Deep Visual-Semantic Model
    \expitem{Combined a trained visual model with an embedded semantic model to allow for classification of unsupervised image labels by taking advantage of contextual similarity.}
  \expend
  
  \projtitle{\href{https://github.com/jth1011}{ECE539 Final Project - Image Supersampling}}{Academic Project}
  \expstart
    \expitem{Compared State-of-the-Art CNN and GAN models to evaluate the advantages of each model.}
    \expitem{Created own supersampling CNN that outperforms general upscaling methods such as bicubic interpolation and gaussian denoising.}
  \expend

%-----------LEADERSHIP-----------------
\section{Leadership}
   Academic Chair, Theta Tau Professional Co-Ed Fraternity, 2020 \\
   High School Math \& Science Tutor, Madison West High School, 2019

%-------------------------------------------
\end{document}
